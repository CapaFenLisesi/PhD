%%%%%%%%%%%%%%%%%%%%%%%%%%%%%%%%%%%%%%%%%%%%%%%%%%%%%%%%%%%%%%%%%%%%%%%%%%%%%%%%%%%%%%%%µ%%%%%%%%%%%
%%%  HDL - Towards a Harmonized Dataset Model  %%%%
%%%%%%%%%%%%%%%%%%%%%%%%%%%%%%%%%%%%%%%%%%%%%%%%%%%%%%%%%%%%%%%%%%%%%%%%%%%%%%%%%%%%%%%%%%%%%%%%%%%%

\documentclass[runningheads,a4paper]{llncs}

\usepackage[utf8]{inputenc}
\usepackage{amssymb}
\setcounter{tocdepth}{3}
\usepackage{graphicx}
\usepackage{tabularx}
\usepackage{url}
\usepackage{listings}
\usepackage{subfigure}
\usepackage{algorithmic}
\usepackage{algorithm}
\usepackage{xcolor}
\usepackage{multirow}


\graphicspath{ {figures/} }

\newcommand{\keywords}[1]{\par\addvspace\baselineskip
\noindent\keywordname\enspace\ignorespaces#1}

% todo macro
\usepackage{color}
\newtheorem{deflda}{Axiom}
\newcommand{\todo}[1]{\noindent\textcolor{red}{{\bf \{TODO}: #1{\bf \}}}}

\colorlet{punct}{red!60!black}
\definecolor{background}{HTML}{FFFFFF}
\definecolor{delim}{RGB}{20,105,176}

% Language Definitions for JSON
\lstdefinelanguage{json}{
    basicstyle=\tiny,
    numbersep=4pt,
    showstringspaces=false,
    breaklines=true,
    frame=lines,
    literate=
      {:}{{{\color{punct}{:}}}}{1}
      {,}{{{\color{punct}{,}}}}{1}
      {[}{{{\color{delim}{[}}}}{1}
      {]}{{{\color{delim}{]}}}}{1},
}

%%%%%%%%%%%%%%%%%%%%%%%%%%%%%%%
%%%  Beginning of document  %%%
%%%%%%%%%%%%%%%%%%%%%%%%%%%%%%%

\begin{document}

% first the title is needed
\title{HDL - Towards a Harmonized Dataset Model}

\author{Ahmad Assaf\inst{1,2}, Rapha\"{e}l Troncy\inst{1} and Aline Senart\inst{2}}

\institute{EURECOM, Sophia Antipolis, France, \email{<firstName.lastName@eurecom.fr>}
  \and SAP Labs France, \email{<firstName.lastName@sap.com>}
}

% a short form should be given in case it is too long for the running head
\titlerunning{HDL - Towards a Harmonized Dataset Model}
%\authorrunning{Assaf, Senart and Troncy}

\maketitle

%%%%%%%%%%%%%%%%%%
%%%  Abstract  %%%
%%%%%%%%%%%%%%%%%%

\begin{abstract}

The Open Data movement triggered an unprecedented amount of data published in a wide range of domains. Governments and corporations around the world are encouraged to publish, share, use and integrate Open Data. There are many areas where we can see the value of Open Data, from transparency and self-empowerment to improving efficiency, effectiveness and decision making. The growing amount of data constitutes the need for rich metadata attached to it. This metadata enables dataset discovery, comprehension and maintenance. Data portals, which are considered to be datasets' access points, present their metadata in various models. In this paper, we propose HDL, a harmonized dataset model based on the analysis of seven prominent dataset models (CKAN, DKAT, Public Open Data, Socrata, VoID, DCAT and Schema.org). We further present use cases that show the benefits of providing rich metadata to enable dataset discovery, search and spam detection.

\keywords{Dataset, Dataset Profile, Metadata, Dataset Model}
\end{abstract}

%%%%%%%%%%%%%%%%%%%%%%%%%
%%%  1. Introduction  %%%
%%%%%%%%%%%%%%%%%%%%%%%%%

\section{Introduction}
\label{sec:introduction}

Open data is the data that can be easily discovered, reused and redistributed by anyone. It can include anything from statistics, geographical data, meteorological data to digitized books from libraries. Open data should have both legal and technical dimensions. It should be placed in the public domain under liberal terms of use with minimal restrictions and should be available in electronic formats that are non-proprietary and machine readable. Open Data has major benefits for citizens, businesses, society and governments. It increases transparency and enable self-empowerment by improving the visibility of previously inaccessible information and allowing citizens to be more informed about policies, public spendings and track activities in the law making processes. Moreover, and despite of the legal issues surrounding Linked Data licenses \cite{nomoneyLOD}, it is still considered a gold mine for organizations who are trying to leverage external data sources in order to produce more informed business decisions \cite{Boyd2011}.

Linked Data publishing best practices~\cite{Bizer:2011:EWG:2075914.2075915} specifies that datasets should contain metadata needed to effectively understand and use them. \textit{Metadata} is structured information that describes, explains, locates, or otherwise makes it easier to retrieve, use, or manage an information resource~\cite{NISO:04}. Having rich metadata helps in:

\begin{itemize}
  \item \textbf{Delaying data entropy}: \textit{Information entropy} is the degradation or loss that limits the information content in raw or metadata. The existence of high quality metadata can counteract information entropy and abstract the raw data complexity and dynamicity leading to  increased dataset longevity~\cite{GTOS}.
  \item \textbf{Enhancing data discovery, exploration and reuse}: In~\cite{Graham:11}, it was found that users are facing difficulties finding and reusing publicly available datasets. Metadata provides an overview of the whole datasets making them more searchable and accessible. High quality metadata can be at times more important than the actual raw data especially when the costs of publishing and maintaining such data is high.
  \item \textbf{Enhancing spam detection}: Detecting spam in public data portals is increasingly difficult  even with security measures like captchas and anti-spam devices. Good dataset metadata quality reflects highly on the quality of its raw data.
\end{itemize}

The value of Open Data is recognized when it is used. To ensure that, publishers need to enable people to find datasets easily. Data portals are specifically designed for this purpose. They make it easy for individuals and organizations to store, publish and discover datasets. The data portals can be be public like DataHub\footnote{\url{http://datahub.io}} and Europe's Public Data\footnote{\url{http://publicdata.eu}} or private like Quandl\footnote{\url{https://quandl.com/}} or Engima\footnote{\url{http://enigma.io/}}. The data available in private portals is of higher quality as it is manually curated but in lesser quantity compared to what is available in public portals. Similarly, in some public data portals, administrators manually review datasets information, validate, correct and attach suitable metadata information.

Data models vary across portals. Surveying the models landscape, we did not find any that offers enough granularity to completely describe complex datasets facilitating search, discovery and recommendation. For example, the DataHub\footnote{\url{http://datahub.io}} uses an extension of the Data Catalog Vocabulary (DCAT)~\cite{Erickson:14:DCV}. This data model prohibits a semantically rich representation of complex datasets like DBpedia\footnote{\url{http://dbpedia.org}} where it has multiple endpoits and thousands of dump files with various contents in several languages~\cite{Brummer:2014:DTS:2660517.2660538}. Moreover, to properly integrate Open Data into business, a dataset should include the following information: i)Access information: The dataset is rendered useless if it does not contain accessible data dumps or queryable endpoints. ii)License information: Businesses are always concerned with the legal implications of using external content. As a result, datasets should include both machine and human readable license information that indicates permissions, copyrights and attributions. iii)Provenance information: Depending on the dataset license, the data might not be legally usable if there are no information describing its authoritative and versioning information. Current models underspecified these main aspects limiting the usability of many datasets.

In this paper, we survey the landscape of existing data portals and dataset models (e.g., CKAN, DKAT, Public Open Data, Socrata, VoID, DCAT and Schema.org). We further analyze these models and suggest a classification for metadata information. Based on this classification, we propose HDL, a harmonized dataset model that addresses the shortcomings of existing dataset models based on analyzing seven prominent dataset models.

The remainder of the paper is structured as follows. In Section~\ref{sec:models}, we present the existing dataset models used by various data portals. In Section~\ref{sec:metadata}, we present our classification for the different metadata information. In Section~\ref{sec:hdl}, we describe our proposed model and suggest a set of best practices to ensure proper metadata presentation and we finally conclude and outline some future work in Section~\ref{sec:conclusion}.

%%%%%%%%%%%%%%%%%%%%%%%%%%%%
%%%  2. Data Portals and Dataset Models  %%%
%%%%%%%%%%%%%%%%%%%%%%%%%%%%

\section{Data Portals and Dataset Models}
\label{sec:models}

There are many data portals hosting a large number of private and public datasets. Each portal present the data based on the model used by the underlying software. In this section, we present the results of our landscape survey of prominent data portals and dataset models.

\subsection{DCAT}

Data Catalog Vocabulary (DCAT), a W3C recommendation established designed to facilitate interoperability between data catalogs published on Web~\cite{Erickson:14:DCV}. The goal behind DCAT is to increase datasets discoverability enabling applications to easily consume metadata coming from multiple sources. Moreover, they foresee that aggregated DCAT metadata can facilitate digital preservation and enable decentralized publishing and federated search.

DCAT is an RDF vocabulary defining three main classes (\texttt{dcat:Catalog}, \texttt{dcat:Dataset} and \texttt{dcat:Distribution}). We are interested in: 1)\texttt{dcat:Dataset} class which is a collection of data that can be available for download in one or more formats and 2)\texttt{dcat:Distribution} class which represents the accessible form of a dataset e.g., RSS feed, REST API, SPARQL endpoint.

\subsection{DCAT-AP}

The DCAT application profile for data portals in Europe (DCAT-AP)\footnote{\url{https://joinup.ec.europa.eu/asset/dcat\_application\_profile/description}} is a specification based on DCAT specified to describe public section datasets in Europe. It defines a minimal set that should be included in a dataset profile by specifying mandatory and optional properties. The main goal behind it is to enable cross-portal search and enhance discoverability. DCAT-AP has been promoted by the Open Data Support\footnote{\url{http://opendatasupport.eu}} to be the standard for describing datasets and catalogs in Europe.

\subsection{ADMS}

The Asset Description Metadata Schema (ADMS)~\cite{ADMS:13:W3C} is also a profile of DCAT. It is used to semantically describe assets. They define an asset broadly as something that can be opened and read using familiar desktop software (e.g., code lists, taxonomies, dictionaries, vocabularies) as opposed to something that needs to be processed like raw data. While DCAT is designed to facilitate interoperability between data catalogs, ADMS is focused on the assets within a catalog.

\subsection{VoID}

VoID~\cite{BoHm:2011:CVD:2030805.2031001} is another RDF vocabulary designed specifically to describe linked RDF datasets and bridge the gap between data publishers and data consumers. In addition to dataset metadata, VoID describes the links between datasets.

VoID defines three main classes (\texttt{void:Dataset}, \texttt{void:Linkset} and \texttt{void:subset}). We are specifically interested in the \texttt{void:Dataset} concept. VoID conceptualizes a dataset with a social dimension. A VoID dataset is a collection of raw data, talking about one or more topics, originates from a certain source or process and accessible on the web.

\subsection{CKAN}

CKAN\footnote{\url{http://ckan.org}} is the world's leading open-source data management system (DMS). It helps users from different domains (national and regional governments, companies and organizations) to easily publish their data through a set of workflows to publish, share, search and manage datasets. CKAN is the portal powering web sites like DataHub\footnote{\url{http://datahub.io}}, Europe's Public Data\footnote{\url{http://publicdata.eu}} and the U.S Government's open data\footnote{\url{http://data.gov}}.

CKAN is a complete catalog system with an integrated data storage and powerful RESTful JSON API. It offers a rich set of visualization tools (e.g., maps, tables, charts) as well as an administration dashboard to monitor datasets usage and statistics. CKAN allows publishing datasets either via an import feature or through a web interface. Relevant metadata describing the dataset and its resources as well as organization related information can be added. A Solr\footnote{\url{http://lucene.apache.org/solr/}} index is built on top of this metadata to enable a search and filtering.

The CKAN data model\footnote{http://docs.ckan.org/en/ckan-1.8/domain-model.html} contains information to describe a set of entities (dataset, resource, group, tag and vocabulary). CKAN keeps the core metadata restricted as a JSON file, but allows for additional information to be added via ``extra'' arbitrary key/value fields. CKAN supports Linked Data and RDF as it provides a complete and functional mapping of its model to Linked Data formats.

\subsection{DKAN}

DKAN\footnote{\url{http://nucivic.com/dkan/}} is a Drupal-based DMS with a full suite of cataloging, publishing and visualization features. Built on Drupal, DKAN can be easily customized and extended. The actual data sets in DKAN can be stored either within DKAN or on external sites. DKAN users are able to explore, search and describe datasets through the web interface or a RESTful API.

The DKAN data model\footnote{\url{http://docs.getdkan.com/dkan-documentation/dkan-developers/dataset-technical-field-reference/}} is very similar to that of CKAN, containing information to describe datasets, resources, groups and tags.

\subsection{Socrata}

Socrata\footnote{\url{http://socrata.com}} is a commercial platform to streamline data publishing, management, analysis and reusing. It empowers users to review, compare, visualize and analyze data in real time. Datasets hosted in Socrata can be accessed using RESTful API that facilitates search and data filtering.

Socrata allows flexible data management by implementing various data governance models and ensuring compliance with metadata schema standards. It also enables administrators to track data usage and consumption through dashboards with real-time reporting. Socrata is very flexible when it comes to customizations. It has a consumer-friendly experience giving users the opportunity to tell their story with data.

Socrata API returns the raw data serialized as JSON instead of the dataset's metadata. As a result, to examine the data model we had to investigate Socrata's API source code\footnote{https://github.com/socrata/soda-java/tree/master/src/main/java/com/socrata/model}. The data model is mainly based to represent tabular data, it covers a set basic core metadata and has good support for geospatial data.

\subsection{Schema.org}

Schema.org\footnote{\url{http://schema.org}} is a collection of schemas used to markup HTML pages with structured data. This structured data allows many applications like search engines to understand the information on Web pages, thus improving the display of search results and making it easier for people to find relevant data.

Schema.org covers many domains. We are specifically interested in the \texttt{Dataset} schema. However, there are many classes and properties that can be used to describe organizations, authors, etc.

\subsection{Project Open Data}

Project Open Data (POD)\footnote{\url{http://project-open-data.cio.gov/}} is an online collection of best practices and case studies to help data publishers. It is a collaborative project that aims to evolve as a community resource to facilitate adoption of open data practices and facilitate collaboration and partnership between both private and public data publishers.

The POD metadata model\footnote{\url{https://project-open-data.cio.gov/v1.1/schema/}} is based on DCAT. Similarily to DCAT-AP, POD additionally defines three types of metadata elements: Required, Required-if(conditionally required) and Expanded (optional). The metadata model is presented in JSON format and encourages publishers to extend their metadata descriptions using elements from the ``Expanded Fields'' list, or from any well-known vocabulary.

%%%%%%%%%%%%%%%%%%%%%%%%%%%%
%%%  3. Dataset Models  %%%
%%%%%%%%%%%%%%%%%%%%%%%%%%%%

\section{Metadata Classification}
\label{sec:metadata}

A standard dataset metadata model should contain information about four sections:

\begin{itemize}
  \item \textbf{Resources}: Distributable parts containing the actual raw data. They can come in various formats (JSON, XML, RDF, etc.) and can be downloaded or accessed directly (REST API, SPARQL endpoint).
  \item \textbf{Tags}: Provide descriptive knowledge on the dataset content and structure. They are used mainly to facilitate search and reuse.
  \item \textbf{Groups}: A dataset can belong to one or more group that share common semantics. A group can be seen as a cluster or a curation of datasets based on shared categories or themes.
  \item \textbf{Organizations}: A dataset can belong to one or more organization controlled by a set of users. Organizations are different from groups as they are not constructed by shared semantics or properties, but solely on their association to a specific administration party.
\end{itemize}

Upon examining the various data models, we grouped the metadata information into four main types. Each section discussed above should contain one or more of these types. For example, resources have general, access, ownership and provenance information while tags have general and provenance information only. The four types are:

\begin{itemize}
\item \textbf{General information}: General information about the dataset. e.g., title, description, ID. This general information is manually filled by the dataset owner. In addition to that, tags and group information is required for classification and enhancing dataset discoverability. This information can be entered manually or inferred modules plugged into the topical profiler.

\item \textbf{Access information}: Information about accessing and using the dataset. This includes the dataset URL, license information i.e., license title and URL and information about the dataset's resources. Each resource has as well a set of attached metadata e.g., resource name, URL, format, size.

\item \textbf{Ownership information}: Information about the ownership of the dataset. e.g., organization details, maintainer details, author. The existence of this information is important to identify the authority on which the generated report and the newly corrected profile will be sent to.

\item \textbf{Provenance information}: Temporal and historical information on the dataset and its resources. For example, creation and update dates, version information, version, etc. Most of this information can be automatically filled and tracked.
\end{itemize}


%%%%%%%%%%%%%%%%%%%%%%%%%%%%%%%%%%%%%%%%%%%
%%%  4. Towards A Harmonised Model  %%%
%%%%%%%%%%%%%%%%%%%%%%%%%%%%%%%%%%%%%%%%%%%

\section{Towards A Harmonised Model}
\label{sec:hdl}

%%%%%%%%%%%%%%%%%%%%%%%%%%%%%%%%%%%%%%%%%%%
%%%  5. Conclusion & Future Work  %%%
%%%%%%%%%%%%%%%%%%%%%%%%%%%%%%%%%%%%%%%%%%%

\section{Conclusion and Future Work}
\label{sec:conclusion}

%%%%%%%%%%%%%%%%%%%%%%%%%%%%%%%%%%%%%%%%%%%
%%%  Acknowledgments  %%%
%%%%%%%%%%%%%%%%%%%%%%%%%%%%%%%%%%%%%%%%%%%

\section*{Acknowledgments}
This research has been partially funded by the European Union's 7th Framework Programme via the project Apps4EU (GA No. 325090).
\vspace{0.5cm}

\bibliographystyle{abbrv}
\nocite{*}
\bibliography{HDL}
\end{document}
