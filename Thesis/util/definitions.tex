\usepackage[left=1.4in,right=1.3in,top=1.1in,bottom=1.1in,includefoot,includehead,headheight=13.6pt]{geometry}

\usepackage[absolute]{textpos}
\usepackage{fancyhdr}
\usepackage[usenames,dvipsnames,table]{xcolor}

% Table of contents for each chapter
\usepackage[nottoc, notlof, notlot]{tocbibind}
\usepackage{minitoc}
\setcounter{minitocdepth}{2}
\mtcindent=15pt

% Glossary / list of abbreviations
\usepackage[intoc]{nomencl}
\renewcommand{\nomname}{List of Abbreviations}
\makenomenclature

%%%%%%%%%%%%%%%%%%%%%%%%%%%%%%%%%%%%%%%%%%%%%%%%%%%%%%%%%%%%%%%%%%
% ToDo Code
%%%%%%%%%%%%%%%%%%%%%%%%%%%%%%%%%%%%%%%%%%%%%%%%%%%%%%%%%%%%%%%%%%
\usepackage{xargs}
\newcommandx{\unsure}[2][1=]{\todo[linecolor=red,backgroundcolor=red!25,bordercolor=red,#1]{#2}}
\newcommandx{\change}[2][1=]{\todo[linecolor=blue,backgroundcolor=blue!25,bordercolor=blue,#1]{#2}}
\newcommandx{\info}[2][1=]{\todo[linecolor=green,backgroundcolor=green!25,bordercolor=green,#1]{#2}}
\newcommandx{\improvement}[2][1=]{\todo[linecolor=magenta,backgroundcolor=magenta!25,bordercolor=magenta,#1]{#2}}
\newcommandx{\fix}[2][1=]{\todo[linecolor=red,backgroundcolor=red!25,bordercolor=red,#1]{#2}}

%%%%%%%%%%%%%%%%%%%%%%%%%%%%%%%%%%%%%%%%%%%%%%%%%%%%%%%%%%%%%%%%%%
% PDF Code
%%%%%%%%%%%%%%%%%%%%%%%%%%%%%%%%%%%%%%%%%%%%%%%%%%%%%%%%%%%%%%%%%%

\usepackage{ifpdf}

\ifpdf
  \usepackage[pdftex]{graphicx}
  \DeclareGraphicsExtensions{.jpg}
  \usepackage{bookmark, hyperref}
\else
  \usepackage{graphicx}
  \DeclareGraphicsExtensions{.ps,.eps}
  \usepackage[dvipdfm]{bookmark, hyperref}
\fi

\graphicspath{{.}{images/}}

% definitions.

\setcounter{secnumdepth}{3}
\setcounter{tocdepth}{2}

% Some useful commands and shortcut for maths:  partial derivative and stuff

\newcommand{\pd}[2]{\frac{\partial #1}{\partial #2}}
\def\abs{\operatorname{abs}}
\def\argmax{\operatornamewithlimits{arg\,max}}
\def\argmin{\operatornamewithlimits{arg\,min}}
\def\diag{\operatorname{Diag}}
\newcommand{\eqRef}[1]{(\ref{#1})}
\renewcommand\contentsname{Table of Contents}

\renewcommand{\texttt}[1]{{\small\ttfamily #1}}
%%%%%%%%%%%%%%%%%%%%%%%%%%%%%%%%%%%%%%%%%%%%%%%%%%%%%%%%%%%%%%%%%%
% Fancy Header Style Options
%%%%%%%%%%%%%%%%%%%%%%%%%%%%%%%%%%%%%%%%%%%%%%%%%%%%%%%%%%%%%%%%%%

% Sets fancy header and footer
\pagestyle{fancy}
% Delete current footer settings
\fancyfoot{}
% Page number (boldface) in left on even
\fancyhead[LE,RO]{\bfseries\thepage}
% pages and right on odd pages
\fancyhead[RE]{\small\bfseries\nouppercase{\leftmark}}      % Chapter in the right on even pages
\fancyhead[LO]{\small\bfseries\nouppercase{\rightmark}}     % Section in the left on odd pages
\let\headruleORIG\headrule
\renewcommand{\headrule}{\color{black} \headruleORIG}
\renewcommand{\headrulewidth}{1.0pt}
\arrayrulecolor{black}

\fancypagestyle{plain}{
  \fancyhead{}
  \fancyfoot{}
  \renewcommand{\headrulewidth}{0pt}
}

% Clear Header Style on the Last Empty Odd pages
\makeatletter

\def\cleardoublepage{\clearpage\if@twoside \ifodd\c@page\else%
  \hbox{}%
  \thispagestyle{empty}%              % Empty header styles
  \newpage%
  \if@twocolumn\hbox{}\newpage\fi\fi\fi}

\makeatother

\fancypagestyle{lscape}{
  \fancyhf{}
  \fancyfoot[LE]{
    \begin{textblock}{10}(1,22){\rotatebox{90}{\small\bfseries\nouppercase\leftmark}}\end{textblock}
    \begin{textblock}{1}(1,1){\rotatebox{90}{\thepage}}\end{textblock}
    \begin{textblock}{10}(1.5,0.4){\rotatebox{90}{\rule{29cm}{1pt}}}\end{textblock}
  }
  \fancyfoot[LO] {
    \begin{textblock}{1}(1,1){\rotatebox{90}{\thepage}}\end{textblock}
    \begin{textblock}{10}(1,22){\rotatebox{90}{\small\bfseries\nouppercase\rightmark}}\end{textblock}
    \begin{textblock}{10}(1.5,0.4){\rotatebox{90}{\rule{29cm}{1pt}}}\end{textblock}
  }
  \renewcommand{\headrulewidth}{0pt}
  \renewcommand{\footrulewidth}{0pt}
}

\fancypagestyle{standard}{
  \fancyhf{}
  % Page number (boldface) in left on even
  \fancyhead[LE,RO]{\bfseries\thepage}
  % pages and right on odd pages
  \fancyhead[RE]{\small\bfseries\nouppercase{\leftmark}}      % Chapter in the right on even pages
  \fancyhead[LO]{\small\bfseries\nouppercase{\rightmark}}     % Section in the left on odd pages
  \renewcommand{\headrule}{\color{black} \headruleORIG}
  \renewcommand{\headrulewidth}{1.0pt}
}

\setlength{\TPHorizModule}{1cm}
\setlength{\TPVertModule}{1cm}
%%%%%%%%%%%%%%%%%%%%%%%%%%%%%%%%%%%%%%%%%%%%%%%%%%%%%%%%%%%%%%%%%%%%%%%%%%%%%%%
% Prints your review date and 'Draft Version' (From Josullvn, CS, CMU)
%%%%%%%%%%%%%%%%%%%%%%%%%%%%%%%%%%%%%%%%%%%%%%%%%%%%%%%%%%%%%%%%%%%%%%%%%%%%%%%
\newcommand{\reviewtimetoday}[2]{\special{!userdict begin
    /bop-hook{gsave 20 710 translate 45 rotate 0.8 setgray
      /Times-Roman findfont 12 scalefont setfont 0 0   moveto (#1) show
      0 -12 moveto (#2) show grestore}def end}}

\newenvironment{maxime}[1]
{
\vspace*{0cm}
\hfill
\begin{minipage}{0.5\textwidth}
\hrulefill $\:$ {\bf #1}\\
\it
}
{

\hrulefill
\vspace*{0.5cm}%
\end{minipage}
}

\let\minitocORIG\minitoc
\renewcommand{\minitoc}{\minitocORIG \vspace{1.5em}}

\newenvironment{bulletList}%
{ \begin{list}%
  {$\bullet$}%
  {\setlength{\labelwidth}{25pt}%
   \setlength{\leftmargin}{30pt}%
   \setlength{\itemsep}{\parsep}}}%
{ \end{list} }

\newtheorem{definition}{D?finition}
\renewcommand{\epsilon}{\varepsilon}

%%%%%%%%%%%%%%%%%%%%%%%%%%%%%%%%%%%%%%%%%%%%%%%%%%%%%%%%%%%%%%%%%%
% Centered page environment
%%%%%%%%%%%%%%%%%%%%%%%%%%%%%%%%%%%%%%%%%%%%%%%%%%%%%%%%%%%%%%%%%%

\newcommand{\Author}{Ahmad A Assaf}
\newcommand{\Subject}{Enabling Self Service Data Provisioning Through Semantic Enrichment of Data}

\newenvironment{vcenterpage}
{\newpage\vspace*{\fill}\thispagestyle{empty}\renewcommand{\headrulewidth}{0pt}}
{\vspace*{\fill}}
\usepackage{listings}
\hypersetup{
    unicode=false,                        % non-Latin characters in Acrobat?s bookmarks
    pdftoolbar=true,                      % show Acrobat?s toolbar?
    pdfmenubar=true,                      % show Acrobat?s menu?
    pdffitwindow=false,                   % window fit to page when opened
    pdfstartview={FitH},                  % fits the width of the page to the window
    plainpages=false,
    pdftitle={My title},                  % title
    pdfauthor={Author},                   % author
    pdfsubject={Subject},                 % subject of the document
    pdfcreator={Author},                  % creator of the document
    pdfnewwindow=true,                    % links in new window
    colorlinks=true,                      % false: boxed links; true: colored links
    linkcolor=ForestGreen,                % color of internal links
    citecolor=red,                        % color of links to bibliography
    filecolor=magenta,                    % color of file links
    urlcolor=blue,                        % color of external links
    % pageanchor=false
}

\newlength{\plarg}
\setlength{\plarg}{16cm}
\newlength{\glarg}
\setlength{\glarg}{17cm}

%%%%%%%%%%%%%%%%%%%%%%%%%%%%%%%%%%%%%%%%%%%%%%%%%%%%%%%%%%%%%%%%%%
% Listings Definitions
%%%%%%%%%%%%%%%%%%%%%%%%%%%%%%%%%%%%%%%%%%%%%%%%%%%%%%%%%%%%%%%%%%
\colorlet{punct}{red!60!black}
\definecolor{background}{gray}{0.97}
\definecolor{delim}{RGB}{20,105,176}

\lstdefinelanguage{json}{
    basicstyle=\small\ttfamily,
    showstringspaces=false,
    breaklines=true,
    frame=lines,
    captionpos=b,
    aboveskip=3mm,
    belowskip=3mm,
    backgroundcolor=\color{white},
    literate=
      {:}{{{\color{punct}{:}}}}{1}
      {,}{{{\color{punct}{,}}}}{1}
      {[}{{{\color{delim}{[}}}}{1}
      {]}{{{\color{delim}{]}}}}{1},
}

\lstdefinestyle{base}{
  basicstyle=\scriptsize,
  emptylines=1,
  breaklines=true,
  moredelim=**[is][\color{red}]{@}{@},
  moredelim=**[is][\color{olive}]{!}{!},
  moredelim=**[is][\color{blue}]{*}{*},
}

\renewcommand{\ttdefault}{pcr}
\lstdefinelanguage{owl} {
  language=xml,
  basicstyle={\footnotesize\ttfamily},
  numbers=none,
  backgroundcolor=\color{white},
  aboveskip=3mm,
  belowskip=3mm,
  showstringspaces=false,
  columns=flexible,
  keywordstyle={\bfseries\color{blue}},
  commentstyle={\color{red}\textit},
  stringstyle=\color{magenta},
  frame=lines,
  breaklines=true,
  breakatwhitespace=true,
  tabsize=4,
  morekeywords={rdf,rdfs,owl},
  moredelim=*[s][\ttfamily]{:}{:} %Newly added line
}

%% define N3 look and feel
\lstdefinelanguage{N3}{
      frame=lines,
      keywordstyle={\bfseries\color{red}},
      commentstyle={\color{blue}\textit},
      stringstyle=\color{magenta},
      morekeywords=[1]{@prefix, a },
      morestring=[b]",
      morecomment=[s]{<}{>}, % missusing comments for URIrefs
      otherkeywords={^, [, ], (, )},%
      sensitive=false%
}[keywords,comments,strings]

\listfiles

\newcommand\CyrGuillemot{%
  \def\selectguillfont{\fontencoding{OT2}\fontfamily{wncyr}\selectfont}
  \def\guillemotleft{\selectguillfont\symbol{60}}
  \def\guillemotright{\selectguillfont\symbol{62}}
}

\newcommand\PlGuillemot{%
  \def\selectguillfont{\fontencoding{OT4}\fontfamily{cmr}\selectfont}
  \def\guillemotleft{\selectguillfont\symbol{174}}
  \def\guillemotright{\selectguillfont\symbol{175}}
}

\newcommand\LaGuillemot{%
  \def\selectguillfont{\fontencoding{U}\fontfamily{lasy}%
    \fontseries{m}\fontshape{n}\selectfont}
  \def\guillemotleft{\selectguillfont\hbox{\symbol{40}%
    \kern-0.20em\symbol{40}}}
  \def\guillemotright{\selectguillfont\hbox{\symbol{41}%
    \kern-0.20em\symbol{41}}}
}

\newcommand\ECGuillemot{%
  \def\selectguillfont{\fontencoding{T1}\fontfamily{cmr}\selectfont}
  \def\guillemotleft{\selectguillfont\symbol{19}}
  \def\guillemotright{\selectguillfont\symbol{20}}
}

\newcommand\LMGuillemot{%
  \def\selectguillfont{\fontencoding{T1}\fontfamily{lmr}\selectfont}
  \def\guillemotleft{\selectguillfont\symbol{19}}
  \def\guillemotright{\selectguillfont\symbol{20}}
}

\newcommand\CyrGLeft{\CyrGuillemot\guillemotleft}
\newcommand\CyrGRight{\CyrGuillemot\guillemotright}
\newcommand\PlGLeft{\PlGuillemot\guillemotleft}
\newcommand\PlGRight{\PlGuillemot\guillemotright}
\newcommand\LaGLeft{\LaGuillemot\guillemotleft}
\newcommand\LaGRight{\LaGuillemot\guillemotright}
\newcommand\ECGLeft{\ECGuillemot\guillemotleft}
\newcommand\ECGRight{\ECGuillemot\guillemotright}
\newcommand\LMGLeft{\LMGuillemot\guillemotleft}
\newcommand\LMGRight{\LMGuillemot\guillemotright}

\newcolumntype{L}{>{\arraybackslash}m{13cm}}

\newcommand{\algorithmicrequire}{\textbf{Require:}}
\newcommand{\algorithmicensure}{\textbf{Ensure:}}
