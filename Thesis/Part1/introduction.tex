\chapter*{Overview of Part~\ref{part:dataset_profiling}}

In Part~\ref{part:dataset_profiling}, we focus on the development of a framework that automatically validates and generates dataset profiles. We highlight the extensibility of this framework and show the results of running it across various data portals.

In Chapter~\ref{chapter:part1-background}, we overview the background of our work in data profiling and quality assurance. We first introduce the basic concepts in the Semantic Web and the important aspects related to (Linked) Open Data. Then, we describe the concepts of data profiling and data quality.

In Chapter~\ref{chapter:hdl}, we conduct a unique and comprehensive survey of seven metadata models: CKAN, DKAN, Public Open Data, Socrata, VoID, DCAT and Schema.org. We propose a Harmonized Dataset modeL (HDL) based on this survey. We describe use cases that show the benefits of providing rich metadata to enable dataset discovery, search and Spam detection.

In Chapter~\ref{chapter:roomba}, we note the need for tools that are able to identify various issues in this metadata and correct them automatically. We introduce Roomba, a scalable automatic approach for extracting, validating, correcting and generating descriptive linked dataset profiles. Afterwards, we present the results of running our framework on prominent data portals and analyze the results.

In Chapter~\ref{chapter:data-quality}, we survey the landscape of Linked Data quality tools and build upon previous efforts with focus on objective data quality measures. We further present a comprehensive objective quality framework applied to the Linked Open Data. We identify several gaps in the current tools and find the need for a comprehensive evaluation and assessment framework for measuring quality on the dataset level. We extend Roomba to calculate 82\% of the suggested datasets objective quality indicators.