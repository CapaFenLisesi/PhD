\chapter*{Conclusion of Part~\ref{part:dataset_profiling}}

In this part, we presented the various parts required to automatically assess and build harmonized dataset profiles.

First, we surveyed the landscape of various models and vocabularies that described datasets on the web. We have identified four main sections that should be included in the model and classified the information to be included into eight types. We proposed HDL, a harmonized dataset model, that takes the best out of these models and extends them to ensure complete metadata coverage to enable data discovery, exploration and reuse.

Second, we detail the gaps in the current tools for automatic validation and generation of dataset profiles. Afterwards, we propose Roomba to tackle these gaps and show the results of running it on various data portals.

Last, we cover the quality dimension from HDL. We propose an objective assessment framework by identifying quality indicators that can be automatically measured by tools. We further survey the landscape of quality tools and discover various shortcomings. As a result, we extend Roomba and cover 82\% of the proposed quality indicators.

Going back to our scenario, our data portal administrator \textbf{Paul} will be able to use HDL as a basis to extend and present the datasets he controls. Moreover, he can use HDL and the proposed mappings as a basis to extend Roomba to support various dataset models like DKAN or Socrata.

Roomba with its quality extension helps \textbf{Paul} to have a detailed overview on the health and quality of the datasets. He can use it to automatically fix some issues, and notify the datasets owners of the other issues to be manually fixed. He will be able to identify spam datasets resulting in higher data quality.

\textbf{Dan} on the other will be able to have access to cleaner, richer set of datasets. He will be able to examine detailed attributes of the datasets. This will help \textbf{Dan} to make more infomred decisions on which dataset to use in his report.



