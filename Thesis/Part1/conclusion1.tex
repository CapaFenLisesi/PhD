\chapter*{Conclusion of Part~\ref{pa:part1}}

In this part, we presented the different steps required to build an event-based environment, harnessing the wealth of information provided by different Web services. We used the Semantic Web technologies for connecting the sparse event and media descriptions, so that they become more discoverable and reusable. Overall, we described how the event-centric data has been extracted, converted, interlinked and published following the Linked Data principles.
\\
\\
First, we developed a framework that offers a simple-to-use and flexible tool to scrap events and related media using some popular Web services. Data is continuously feeding EventMedia, a RDF dataset published in the Linked Data cloud. 
\\
\\
Second, we detailed the challenges faced to reconcile data retrieved from heterogeneous sources. Evaluations results show how the event matching is sensitive to the temporal distance, and how an efficient string similarity improves the accuracy. Finally, we tackled the problem of linking microposts with fine-grained events, which represents a tremendous challenge given the extreme noise in media content. An important characteristic has driven the design of our approaches which is the real-time nature of events.


