\chapter{Conclusions and Future Perspectives}  \label{ch:conclusion}

In this chapter, we summarize the major achievements of this thesis and we give an outlook on future perspectives.

\section{Achievements}

An ever increasing amount of information spread on the Web is centered on the notion of ``event''. Currently, most of companies that provide calendar of events such as Eventful, Last.fm and Lanyrd are using Web 2.0. They provide an environment where users can view and create an event, and locate events through keyword-based search and ranked results. Such design as unconnected data silos is, however, different from the conception to which aim the ``Web of events''. There is no support to handle the natural relationships that link events at different levels (e.g. similarity) or to link events with their experiential attributes such as discussions and captured media. Indeed, associating events with background knowledge and media ,and linking events together may change the way people or systems exploit data.

This thesis thoroughly describes the different steps aiming to realize the vision of the Web of events having as a foundation the Social Web and harnessing the Semantic Web technologies. The work presented does not focus on the building process, but also on reusing the new representation of events in various areas including Web applications and personalization. The contributions made are:

\begin{itemize}
\item \textbf{Data Structuring} 
\vspace{1mm}
\\
A milestone towards the Web of events is to semantically model what an ``event'' is. Due to its  inherently multidimensional nature, we surveyed some different definitions, and we retained the one which represents the most described aspects. This definition is based on the \textit{Ws} questions: \textit{What}, \textit{When}, \textit{Where} and \textit{Who}. To formalize the event definition, we opted for the LODE ontology as an interoperable model realized without any particular interpretation or perspective. This complies with our strategy to retrieve and model any type of event from the Social Web. On the other hand, the modeling of media was simply achieved by the reuse of popular ontologies in the domain.

\item \textbf{Data Aggregation}
\vspace{2mm} 
\\
Many Web directories contain event-centric data including calendar of events or captured media. Aggregating this data and exploring the explicit overlap between these directories are parts of the building process. Thus, we developed a framework that collects events and media, and exploits explicit metadata (e.g. machine tags, hashtag) to link them. We followed one design requirement which is the \textit{flexibility}. The objective is to be able to flexibly add more event and media directories in the future.

\item \textbf{Data Reconciliation}
\vspace{2mm}  
\\
We particularly addressed two different tasks having in common the challenge of data heterogeneity. The former creates identity link between two same real-world instances and the latter aligns events with microposts at the sub-event level of granularity. For the first task, we surveyed existing automatic instance matching tools. Yet, none of them is able to overcome the heterogeneity found between event directories. As a solution, we proposed an approach based on the correlation and the coverage of predicates and taking into account various data types. As for the second task, we proposed a Named Entity-based approach to bridge the gap between the unstructured content of microposts and the structured description of events. The idea is to exploit the mapping between the classification of Named Entities and the concepts of the event ontology. Both tasks ensure a real-time reconciliation to face the dynamics of events.

\item \textbf{Application and Analysis}
\vspace{2mm}  
\\ We developed some Web Applications enabling new mechanisms to browse and search for events or to create events in a controllable way. Our experience highlighted some limitations to use RDF data within conventional Web technologies. We also showed the importance to design a simple data model  at the expense of expressivity. Lastly, we explored the benefits of Linked Data to uncover behavioral aspects and to improve the user profiling. 

\item \textbf{Event Recommendation}
\vspace{2mm}  
\\We designed a hybrid recommender system in order to suggest personalized events. It is built on top of Semantic Web and combines content-based recommendation and collaborative filtering. It is shown that ontology-enabled feature extraction and enrichment with Linked Data significantly improve the performance. In addition, a user may be involved in many events, but interested in specific topics. Thus, we proposed a method to alleviate the impact of the topical diversity that may characterize a user profile. Results underlined the importance of the social information and the user interests modeling in event recommendation.
\item \textbf{Community Detection in EBSN}
\vspace{2mm}  
\\We presented an approach to detect topical communities in event-based social network relying on structural and content features. The links between events and media were used to construct event-based networks from media directories. We also built networks using the co-attendance information obtained from event directories. The evaluation results shed light on the difference between these networks in terms of users' interactions.
\end{itemize}

In summary, the contributions achieved pave the way to build the Web of events as part of Linked Data. The main idea is to bring together event-centric data into a unified structured knowledge with the flexibility and depth afforded by the Semantic Web technologies. As rich data made available in the Linked Data cloud, one can expect efficient supports to browse, search and visualize rich data. The work presented in this thesis goes beyond this fact and further demonstrates the utility of the Semantic Web in other tasks such as personalization, user modeling or comparative analysis. Although focused on events, some proposed approaches could be easily propagated to other domains such as movie recommendation or community detection in social media.

\section{Perspectives}
A growing number of RDF datasets continuously feed the Linked Data cloud covering a multitude of diverse domains. This thesis specifically targeted the event domain to build and leverage a meaningful knowledge base. Still, it could be extended by the following future directions:

\begin{itemize}
\item \textbf{Enrichment}
\vspace{1mm}
\\One enhancement is to enrich EventMedia dataset with other popular Web services such as Facebook and Eventbrite\footnote{\url{http://www.eventbrite.com}}. As such, we can increase the overlap in terms of coverage and benefit the assets of each website. Indeed, at the time of writing, the integration of Facebook was under development. Enrichment could also improve data modeling by incorporating useful vocabularies such as the Tickets ontology~\cite{Hepp:2010} or describing participants' friendships. 


\item \textbf{Connection of Events}
\vspace{1mm}
\\ Events sharing spatial-temporal context or having in common a specific topic or participants may have a connection between them. This reveals a key aspect in the Web of events which is to represent the natural relationships at different levels such as referential, structural and causal. While we only dealt with identity connection, the other types remain unexplored. This opens the door for future work exploring more meaningful connections. Moreover, the existing approaches mostly address a specific domain (e.g. historical~\cite{Corda:IESD12}) and focus on specific event attributes (e.g. time~\cite{Vikramaditya:ICTA07}). There is a need for a formal specification that takes into account all the event attributes proving its efficiency to be applied in different domains (e.g. social, political). 

\item \textbf{Temporal Dynamics}
\vspace{1mm}
\\
Temporal dynamics is an important aspect that recently drives the way to design computing applications. The growth of online activities has led to new challenges about how to handle streaming data, instead of static files, which needs more efficiency and scalability. Moreover, data may shift over time, a fact that may impact many tasks such as reconciliation or recommendation. In instance matching, solving at the same time the high heterogeneity and temporal dynamics is a quite challenging problem. Our strategy focused on the heterogeneity problem still need a ground truth to learn the correlation and the coverage of properties. In order to face a future evolution, one solution that can be sought is to automatically generate a ground truth or to fully rely on an unsupervised method. Temporal dynamics has also an impact on event recommendation. Unlike a classic product, an event is ephemeral, and as such, the list of events in the profile of a very active user become unmanageable. To solve this, one simple approach commonly used is to discard irrelevant instances using a time window~\cite{Tsymbal:04}. In our scenario, one can run SPARQL queries to simply index recent events, which raises the question about the effective window size and its impact on the system performance. 

\item \textbf{Scalable Recommender System}
\vspace{1mm}
\\The information overload is a well-known problem that prevents users from easily making decision in an online service even when supporting browsing and searching capabilities. To overcome this problem in event-based service, we proposed a hybrid recommender system based on the classical Vector Space Model (VSM). Although efficient to provide personalized events, our approach has a serious drawback of scalability since the time complexity is linear to the number of events (i.e. documents in VSM). Considering this limitation, several optimization techniques found in the literature could be integrated to speed up the computation. One technique is to reduce complexity in VSM by pruning unnecessary similarity comparisons. This can be ensured by the high-dimensional similarity search techniques such as the popular indexing method named Locality Sensitive Hashing (LSH)~\cite{Gionis:VLDB99}. Another solution worth to be investigated is the multi-relational learning using tensor factorization which can be applied in Linked Data. This is particularly the goal of Rescal-ALS, a scalable tool that represents entities in a latent space enabling efficient information propagation via the dependency structure~\cite{Nickel:ECML13}.

\item \textbf{Community-based Recommendation}
\vspace{1mm}
\\Exploiting community detection for recommendation has been the subject of numerous research studies. It is also an indirect way to assess the quality of the identified communities. Indeed, it has been shown that taking advantage from a collective behavior of users is one solution to alleviate the cold-start problem~\cite{Shaghayegh:RSWEB11,Bessa:AMW12} or to diversify recommendation~\cite{Fatemi:ICSC13}. In this perspective, our recommender system could be improved by the integration of community detection approach applied on event-based social network. Another similar direction is to build a signed network from user interactions as has been proposed by Maniu et al.~\cite{Maniu:2011}, which can be used to build a trust-aware recommender system.

\end{itemize}
