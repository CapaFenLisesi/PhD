\chapter{Installation and Cutomization Instructions}
\label{appendix:appendixD}

\section{Installation and cutomization instructions for Roomba}
\label{section:installation_roomba}

You can either download Roomba from the Github repository\footnote{\url{https://github.com/ahmadassaf/opendata-checker}} as a zip file or clone directly through git. Pay attention when cloning as there is a submodule defined and has to be cloned recursively as well. This can be done via:

\begin{verbatim}
git clone --recursive http://github.com/ahmadassaf/opendata-checker
\end{verbatim}

If you have cloned without \texttt{--recursive}, you may find out that some folders are empty. To fix this:
\begin{verbatim}
git submodule update --init
\end{verbatim}

After successfully having cloned Roomba to your local machine.
\begin{itemize}
	\item Install dependencies by running the command "npm install"
	\item Start Roomba by running "node DC.js"
\end{itemize}

\section{Customizing Roomba}

There are a set of options that you can customize. They can be edited from \texttt{options.json} file (see Listing~\ref{roombaOptions})

\lstset{basicstyle=\scriptsize, backgroundcolor=\color{white}, frame=single, caption= {Roomba's customization via the options file}, label=roombaOptions, captionpos=b}
\lstinputlisting[breaklines=true,language=json]{util/attachments/roomba_options.json}

\begin{itemize}
	\item \textbf{locale}: the language of the messages and the prompts. Default: \textbf{en}
	\item \textbf{cacheFolderName}: the name of cache folder separated by /. Default: \textbf{/cache/}
	\item \textbf{licensesFolder}: the location of the Open Licenses repository\footnote{\url{https://github.com/okfn/licenses}}. It is defined as a submodule and by default it is located in \texttt{/util/licenses/}.
	\item \textbf{mappingFileName}: the name of manual license mappings. Default: \textbf{licenseMappings}
	\item \textbf{proxy}: proxy server e.g., \texttt{proxy:8080}. Default: \textbf{""}
\end{itemize}

\subsection{Localization}

If you wish to translate the messages and prompts into other languages than English, you have to create a new language entry in the \texttt{util/messages.js} with the new locale code e.g., \texttt{fr}. Afterwards, you should keep the object keys intact by translating the values into the desired language. See Listing~\ref{roombaLocalization} for an example.

\lstset{basicstyle=\scriptsize, backgroundcolor=\color{white}, frame=single, caption= {Roomba's localization file example}, label=roombaLocalization, captionpos=b}
\lstinputlisting[breaklines=true,language=json]{util/attachments/roomba_locale.json}

\section{Installation and cutomization instructions for Knowledge Graph Scraper}
\label{section:installation_KGB}

You can either download the knowledge graph scrapper from the Github repository\footnote{\url{https://github.com/ahmadassaf/kbe}} as a zip file or clone directly through git.

\begin{verbatim}
git clone http://github.com/ahmadassaf/kbe
\end{verbatim}

After successfully having cloned the repository to your local machine.
\begin{itemize}
	\item Install dependencies by running the command "npm install"
	\item Start Roomba by running "node KBE.js"
\end{itemize}

The script will automatically create all the required Cache folders:

\begin{itemize}
	\item Main cache folder called \textbf{cache} in the root folder of the application.
	\item \textbf{GKB} inside the cache folder which will hold the aggregated Google Knowledge boxes extracted for a DBpedia concept.
	\item \textbf{instances\_GKB} inside the cache folder which will hold the Google Knowledge box for a single instance.
	\item \textbf{instances} inside the cache folder which will hold the DBpedia instances for each concept.
	\item \textbf{instance\_properties} inside the cache folder which will hold the distinct list of properties for all the instances of a certain concept.
\end{itemize}

The application is run in the console and the output will be available in \texttt{/cache/result.json}.

\subsection{Customization}

There are a set of options that you can customize. They can be edited from \texttt{options.json} file (see Listing~\ref{customizeKBE})

\lstset{basicstyle=\scriptsize, backgroundcolor=\color{white}, frame=single, caption= {Roomba's localization file example}, label=customizeKBE, captionpos=b}
\lstinputlisting[breaklines=true,language=json]{util/attachments/kbe.json}

\begin{itemize}
	\item \textbf{cache\_dbpedia\_concepts}: caches the concepts retrieved from
	  DBpedia.
	\item \textbf{limit\_dbpedia\_concepts}: limits the number of concepts
	  retrieved by DBpedia, false will retrieve all the concepts.
	\item \textbf{limit\_dbpedia\_instances}: limits the number of instances
	  retrieved for each concept, false will retrieve all the instances.
	\item \textbf{limit\_dbpedia\_concepts\_value}: specifies the number of concepts that you wish to retrieve.
	\item \textbf{limit\_dbpedia\_instances\_value}: specifies the number of instancesthat  you wish to retrieve for each concept.
	\item \textbf{proxy}: specifies the proxy address string containing ports e.g., \texttt{proxy:8080}. Default: \textbf{""}.
\end{itemize}

Since Google changes the CSS selectors dynamically at random times, the user can always check the corresponding CSS class name selectors for the Google Knowledge Panel and edit them if needed in the same \texttt{options.json} file as shown in Listing~\ref{customizeKBE_CSS}.

\lstset{basicstyle=\scriptsize, backgroundcolor=\color{white}, frame=single, caption= {Roomba's localization file example}, label=customizeKBE_CSS, captionpos=b}
\lstinputlisting[breaklines=true,language=json]{util/attachments/kbe_css.json}