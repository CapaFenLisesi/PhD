\chapter{Introduction}  \label{ch:intro}
Services such as event directories, social networks and media platforms host an ever increasing amount of event-centric data. Recently, they have attracted people to organize and distribute their personal data according to occurring events, to share related media and to create new social connections. Still, this data needs to be structured and integrated in order to enhance different tasks such as content presentation, recommendation and social analysis. 

\section{Context and Motivation}      \label{sec:motivation}
Roughly speaking, ``\emph{event}'' is a phenomena that has happened or scheduled to happen at a specific place and time. According to recent studies in neuroscience~\cite{Zacks:01}, event is also considered as past experience with which humans remember their real life. A common practice for humans is to naturally organize their personal data according to occurring events: wedding, conference, concert, party, etc. They would like to plan activities according to future events or to record what happened during past events.

Along with the emergence of Web 2.0, people become more involved in online activities sharing rich content to describe events and engaging in social interactions. This is reflected in many social services where a large amount of data exists in multiple modalities such as event attributes (e.g. time, location) and explicit RSVP (i.e. expressing the user intent to join social events) in event directories (e.g. Eventful, Last.fm, Lanyrd, Facebook), photos and videos captured during events and shared on media platforms (e.g. Flickr, YouTube), digital chatter generated by reactions to events in social network sites (e.g. Twitter, Facebook). Yet, this knowledge forms a huge space of disconnected data fragments providing limited event coverage~\cite{Fialho:EVENTS10}. For instance, while Last.fm sustains a broad coverage on event attendance, other valuable details are often missing such as description, price and media. Users tend to use other channels to complement the event overview. Moreover, most of event directories provide limited browsing options (e.g. lack of location map) and unreliable event recommendation (e.g. no consideration of like-minded users). These limitations have been notably highlighted in an exploratory user centered study conducted to assess the perceived benefits and drawbacks of event websites~\cite{Troncy:COLD10}. Having in mind the findings of this study, we focus on two major tasks which are data reconciliation and personalization.

\subsection{Data Reconciliation}

A large amount of event-centric data is spread across multiple services, however, often incomplete and always locked into the sites.
How to leverage the wealth of this information is a serious challenge towards providing a broad coverage of events. As a solution, integrating data is a prominent way to deliver more complete and accurate information. In particular, the recent use of Semantic Web technologies proves to be effective to ensure a large-scale and flexible data integration. With ontologies, developers can structure large amounts of heterogeneous data independently of particular applications, while explicitly representing enriched semantics. In order to deliver enriched event views and higher information coverage, two ways can be explored in data reconciliation with a focus on Semantic Web technologies. 

The Semantic Web is predicated on the availability of large amounts of structured data as RDF, not in isolated islands, but as a Web of interlinked datasets. Linked Data\footnote{\url{http://linkeddata.org/}} is an ongoing project pursuing this avenue and connecting related data through RDF triples\cite{Bizer:HB09}. It interlinks RDF datasets on a large scale and follows the principles\footnote{\url{http://www.w3.org/DesignIssues/LinkedData.html}} outlined by Tim Berners-Lee in 2006. A fundamental concern in the Semantic Web is the comparison and matching of structured data to achieve the vision of the Linked Data. However, data sources often do not share commonly accepted identifiers (i.e. ISBN codes) and make use of different vocabularies. In particular, data reconciliation has recently gained importance in the Semantic Web community and it comprises two main sub-tasks: the former is the ontology matching which refers to the process of determining correspondences between ontological concepts; the latter is the instance matching which refers to the process of determining correspondences between individuals. In this thesis, we focus on the instance matching task to discover identical individuals referring to the same real-world entity. Indeed, the availability of events provided by disparate Web services increases not only the amount of data, but also the variety of representations of a single event-centric entity (e.g. location, participant, artists, etc.). It has been shown that the reconciled data are more advantageous to enhance data quality by improving both completeness and accuracy~\cite{Naumann:2010}. For example, one data source may contain events with few details about involved artists. Another data source may complement the description by providing the biography with complete discography of those artists.

The second solution is to associate events with user-contributed social media. In fact, real-world events often trigger a tremendous activity on numerous social media platforms. Participants share captured photos and videos during events, tweet status messages and engage in discussions with comments. To mine the intrinsic relationships between events and media, most of existing studies focus on event detection from user-generated content that describes breaking news or social events~\cite{Liu:2011,Becker:WebDB09,Sakaki:WWW10}. Automatic event detection is essentially a clustering problem aiming to group together media documents discussing the same event. Other existing works study this problem within the field of data reconciliation~\cite{Rowe:SWJ12,Xin:SIGMOD05}. The idea behind is to compare instances of different ontological classes (e.g. event class and media class) using their related features such as named entities and contextual information. In this thesis, we exploit this idea and we bridge the gap between structured events and unstructured media data. 

Reconciling event-centric entities or aligning events with media have in common some challenges that originate from the use of online, heterogeneous and distributed sources. First, the same real-world entity is often represented in different ways across the disparate data sources. Some of these entities may be related with short descriptions and featuring noisy information. Moreover, the user-generated content exists typically at large scale and evolves dynamically providing daily a significant amount of events, locations, media, etc. These challenges demand a scalable, real-time and efficient techniques to integrate data.

\subsection{Personalization}

Personalization in online social services have gained momentum over the recent past years. Providing assistance to make decision and select reliable products become part of primary concerns in the e-service area. More specifically, integrating personalization techniques in event-based services is a key advantage to attract people to attend relevant events. Such techniques recently start to draw attention as has been attested by the VP (Vice President) Operations of Eventful reporting that ``\textit{When we really got serious about personalization, we started talking about it a few years ago and we really got busy a couple of years ago}''\footnote{Paul Ramirez, MarketingSherpa Email Summit 2014.}.

One personalization technique is to build a recommender system that decodes the user interests and optimizes accordingly the information perceived. To help such system predict items of interest, various clues are available ranging from the user profile, explicit ratings, to past activities and social interactions. Different from a classic item, event occurs at a specific place and during a period of time to become worthless for recommendation. While the classic items continuously receive useful feedback, the user preferences related to events are very sparse. This problem is mainly due to the transient nature of events leading to the fact that most of users are associated with very limited number of events. Given this high sparsity level, traditional recommender systems fail to handle event recommendation where both content and social information need to be considered~\cite{Cornelis:IICAI05}.  

Another innovative technique is to position the user within one or more communities, instead of an isolated individual~\cite{Paliouras:2012}. In order to enable community-driven personalization, the system needs to analyze networked data and reveal the underlying communities. This demands an efficient method to detect meaningful communities which in turn can benefit various tasks such as customer segmentation, recommendation and influence analysis. In research, several studies has been devoted to solve this problem, but mostly focused on the linkage structure of the network. They assume that the proximity of users is solely reflected by their interactions strength. However, such methods do not consider the topical dimension and often group users having different interests. This problem becomes important when a user interacts with different social objects (e.g. events) inducing highly diverse topics in his/her profile. Consequently, there is a need to incorporate the semantic information along with the linkage structure for detecting meaningful and overlapping communities~\cite{Juan:cason11,Zhongying:12}. 

In this thesis, we tackle the problems related to event recommendation and to community detection in event-based social network. The challenge is to deal with the complex nature of events where social and content information are both important.

\section{Thesis Contributions}      \label{sec:contributions}
As a multidimensional, ephemeral and social entity, the notion of ``\emph{event}'' poses new challenges for research community. In this thesis, we propose approaches related to data reconciliation and personalization in event domain. In summary, the main contributions of this work are as follows:

\begin{itemize}
  \item We created a framework in order to aggregate in real-time event-centric data retrieved from heterogeneous sources. Our strategy is to build an architecture flexible enough to accommodate ongoing growth. Such flexibility is ensured by the ease to add new sources and the use of Semantic Web technologies. The data, continuously collected in real-time, is converted to RDF using existing vocabularies and then stored in a triple store. The entire dataset is called EventMedia.

  \item We propose heuristics to mine the intrinsic connections of event-centric data derived from event directories, media platforms and Linked Data. Given the dynamics of social services, our approach ensures a real-time reconciliation maintaining a dynamic content enhancement. First, we propose a domain-independent reconciliation approach that identifies identical entities residing at heterogeneous sources. Then, we tackle the problem of aligning structured events with unstructured media items based on Natural Language Processing (NLP) techniques.
  
  \item  We consumed Linked Data in order to develop friendly Web applications that meet the user needs: relive experiences based on background knowledge and help create events with consistent details. Then, we highlight the benefits of Linked Data to steer the behavioral analysis and to improve the user profiling.
  
  \item We propose a hybrid system to recommend events based on content features and collaborative participation. This system enriches an event profile with Linked Data and exploits the ontology-enabled feature extraction. It is also enhanced by an approach that detects the effective user interests within a topically diverse user profile. 
      
  \item We introduce a novel approach that detects topical communities within event-based social networks. We distinguish between online and offline networks constructed based on the collaborative participation in events.  Our approach exploits the hierarchical clustering with the combination of both the content features and the the linkage structure. Then, a link-based function is defined to determine the effective user attachment to each community.
    
\end{itemize}

\section{Thesis Outline}      \label{sec:structure}
The work presented in this thesis first describes how to integrate event-centric data into a Semantic Web dataset. Then, it focuses on consuming Linked Data in event domain for the development of Web applications and personalization methods. 
\\
\\
\noindent \textbf{Chapter~\ref{ch:background}} is dedicated to overview the background of our work including the research in event domain and some paradigms related to Semantic Web. We first introduce the important aspects related to events and the basic concepts in the Semantic Web. Then, we describe the evaluation criterion used throughout this work. The rest of this manuscript is composed of two major parts:

\begin{enumerate}
\item In the first part, we focus on the building task that retrieves event-centric data from distributed sources and integrates them into one semantic knowledge base. Such task includes crawling, structuring and linking data, which needs to be ensured with the flexibility afforded by the Semantic Web technologies. The contributions of this part have been published in~\cite{Khrouf:OM11,Khrouf:SWJ12,Khrouf:JWS2013,Khrouf:RAMSS12}.

\begin{itemize}
\item \textbf{Chapter~\ref{ch:data-aggregation}} describes how data is extracted, structured and published following the best practices of the Semantic Web. In particular, we pay attention to create a flexible framework that performs those tasks, and eases the addition of event and media Web services.

\item \textbf{Chapter~\ref{ch:data-reconciliation}} studies the problem of data reconciliation in a heterogeneous environment. We present our approach to detect identical entities in event-centric data by the use of instance matching techniques. Then, we propose a NER-based approach to align events with microposts, thus bridging the gap between structured and unstructured content.

\end{itemize}

\item In the second part, we exploit the constructed knowledge base for various applications. The goal is to highlight the benefits of linked data to improve the event presentation and to explore solutions for personalization in event-based services. The contributions of this part have been published in~\cite{Khrouf:ISWC11,Khrouf:ISWC12,Khrouf:ESWC12,Khrouf:RecSys2013}.

\begin{itemize}

\item \textbf{Chapter~\ref{ch:web-app}} presents three Web applications in charge to support better visualization and to help users search, browse and create events. Besides, it underlines the benefits of our knowledge base, as part of Linked Data, to understand some facts about the user behavior and to improve the user profiling.

\item \textbf{Chapter~\ref{ch:recommendation}} presents our approach built on top of Semantic Web to recommend social events. The idea is to leverage structured and expressive representation of events to predict what a user likes. Our approach is then augmented by the recommendation based on collaborative filtering.

\item \textbf{Chapter~\ref{ch:community-detection}} exploits event-centric users activities in order to construct event-based social networks in online and offline worlds. Then, we propose an approach to detect meaningful communities taking into account the event topics and the linkage structure of the network.
\end{itemize}

\end{enumerate}

\textbf{Chapter~\ref{ch:conclusion}} concludes the presented work and outlines new research directions.
