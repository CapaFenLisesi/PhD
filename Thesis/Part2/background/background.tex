\chapter{Background} \label{chapter:part2-background}
\graphicspath{{Part2/Background/figures/}}

\section{Data Integration}\label{section:dataIntegration}

\textit{Data Integration (DI)} is the process of providing the user with a unified view of data residing at different sources~\cite{Lenzerini:DIT:02}. Data Integration is a challenging task since these sources are in many real-world mutually inconsistent.

Various approaches and methodologies have been proposed to solve the DI problem in the enterprise:
\begin{itemize}
	\item XML as a hierarchical data format can be used as a uniform standard uniform for data representation. However, extending XML to provide complex mappings and sources descriptions is difficult.
	\item SOA can be seen as a holistic approach for distributed systems communication and architecture. In its core, SOA aims at minimizing impedance in the architecture paving the way for easier communication between data sources. However, in~\cite{Frischmuth:ISWC:13}, the authors argue that SOA is well-suited for transaction process rather than an approach for data integration.
	\item Ontologies can be used as a rich format to describe queries and data mappings between schemas and sources. However, developing ontologies require specific skills and it is difficult to provide a complete model that captures the dynamics of the enterprise.
	\item Linked Data paradigm is a slightly different approach from the ontology-based by exploiting Semantic Web technologies like RDF to represent enterprise taxonomies. The LD approach allows terms to be easily reused and extended.
\end{itemize}

\section{Business Intelligence}\label{section:businessIntelligence}

\textit{Business Intelligence (BI)} is the set of techniques and tools for transofmring raw data into meaningful and useful information to be used in the decision making process~\cite{Rud:Wiley:09}. BI included data integration, data quality and data warehousing among others.