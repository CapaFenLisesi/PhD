\chapter*{Conclusion of Part~\ref{pa:part2}}


This part has been devoted to put in use the Linked Data in event domain. Of a particular interest is the applications that handle better event presentation and discovery, as well as the personalization techniques.
\\
\\
Semantic Web applications have been developed to support end-users browsing and creating events. Overall, consuming Linked Data is advantageous to deliver enriched views of events, and to uncover interesting behavioral facts. Still, it was challenging to use conventional Web technologies on top of RDF data, a fact which reminds the trade-off between simplicity and expressivity.
\\
\\
Semantic Web technologies have been exploited to recommend events. Ontology-enabled feature extraction showed its ability to reduce the data sparsity, a problem from which suffers the traditional recommender systems. We highlighted that Linked Data is also beneficial to enrich data, thus improving the performance of our hybrid recommender system.
\\
\\
Finally, we proposed an approach to detect topical communities in event-based social network (EBSN) based on the content and link information. Linking events with media was particularly useful to construct EBSNs from media services. Evaluation shows how people interact differently from one service to another and depending on the event context.